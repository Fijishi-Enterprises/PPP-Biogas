%% Next 2 lines needed for non-Sweave vignettes
%\VignetteEngine{knitr::knitr} 
%\VignetteIndexEntry{Calculating methane and biogas production and production rates using volumetric methods}

\documentclass{article}

%%\usepackage[version=3]{mhchem} %chemical formulas
\usepackage[colorlinks = true, urlcolor = blue]{hyperref} % Must be loaded as the last package

\begin{Schunk}
\begin{Sinput}
> library(knitr)
> #opts_chunk$set(cache=FALSE,tidy=FALSE,highlight=FALSE)
> opts_chunk$set(cache = FALSE, tidy = FALSE, fig.align = "center")
> library(biogas)
>   options(width=75)
\end{Sinput}
\end{Schunk}

\title{Calculating methane and biogas production and production rates using manometric methods}
\author{Nanna Løjborg and Sasha D. Hafner (\texttt{sasha.hafner@eng.au.dk})}

\usepackage{Sweave}
\begin{document}
\Sconcordance{concordance:cumBgMan_function.tex:cumBgMan_function.Rnw:%
1 8 1 1 2 1 0 1 2 1 0 2 1 3 0 1 2 4 1 1 0 83 1}


\maketitle

\section{Introduction}
Biochemical methane potential (BMP) has become an important number in the biogas industri, as it can reveal essential knowledge of several factors of concern when producing biogas, such as substrate and inoculum behavior, or more process related variables (temp., pres., stirring, etc.). It is commonly used to determine the methane potential and anaerobic biodegradability of a given substrate. 
BMP is most commonly evaluated by monitoring the production of biogas. This can be done in several ways including volumetric, manometric, gravimetric and gas density methods. In volumetric methods biogas volumes are measured under constant conditions from different techniques such as.... NTS: add techniques. 

We developed a function that evaluate pressure measurements only, to simplify functions within the biogas package, making it serviceable for public less experinced users. 
This document provides a brief description of the manometric biogas calculation function (cumBgMan) for new users.
We have assumed that readers are familiar with biogas data collection, the biogas package and R.

\section{Overview of the function}
cumBgMan() is a ''high-level'' function within the biogas package. The purpose of cumBgMan() is to convert pressure data collected in the laboratory to cumulative biogas and CH$_4$ production and to calculate production rates. The function can handle data from any number of bottles. For simple operations (e.g. interpolation and standardization of biogas volume) cumBgMan() is supported by calls to external low-level functions (Table \ref{tab:externalfunctionsummary}). The low-level functions are straight-forward to use, and details can be found in their individual help files.
This document describes the use of cumBgMan(). 

\begin{table}[h!]
  \begin{center}
  \caption{Operations done with the low-level functions in cumBgVol(). All functions are vectorized. See help files for more details.}
  \label{tab:lowfunctionsummary}
  \vspace{3pt}
  
  \begin{tabular}{ll}
    \hline
    Operation                                    &   Function \\
    \hline
    Standardise gas volume                       &   \texttt{stdVol()} \\
    Interpolate composition etc.                 &   \texttt{interp()} \\
    Structurize and sort data                    &   \texttt{cumBgDataPrep()} \\   
		\hline
  \end{tabular}
  \end{center}
\end{table}

In general, cumBg* functions are compiled of four sections: check arguments, restructuring and sorting data, interpolation if needed, and biogas standardization and calculations. Restructuring and sorting of data and interpolation are handled by the external functions interp() and cumBgDataPrep(), respectively. From interp() gas composition, cumulative biogas production, and other variables can be interpolated to a specified time if required. From cumBgDataPrep() 'wide' and 'long' data structure are restructured to 'longcombo' data, which is required for cumBgMan() to further calculate cumulative biogas and CH$_4$ production and production rates. Additionally, data is sorted, headspace is added if provided, and composition data is corrected if it seems to be a percentage. Subsequently, the now restructured and sorted data is standardized in cumBgVol() by an external function called stdVol(). 

Two methods are commonly used to evaluate biogas pressure measurements. Method 1 is based on normalized CH4 concentrations, whereas method 2 accounts for the actual CH4 in the bottle headspace. Both methods are available through cumBgMan() and results is expected to be independent of method. The examples below describe cumulative biogas calculation on two dataset having long and longcombo structures, using manometric method 1 and 2, respectibely 
All external functions are within the biogas package

\newline
The arguments for the function are:

% Next line won't wrap correctly. Width doesn't affect args(). Done manually below by copying and editing output in .tex file.
\begin{Schunk}
\begin{Sinput}
>   args(cumBg)
\end{Sinput}
\begin{Soutput}
function (dat, dat.type = "vol", comp = NULL, temp = NULL, pres = NULL, 
    interval = TRUE, data.struct = "long", id.name = "id", time.name = "time", 
    dat.name = dat.type, comp.name = "xCH4", pres.resid = NULL, 
    temp.init = NULL, pres.init = NULL, rh.resid.init = 1, headspace = NULL, 
    vol.hs.name = "vol.hs", headcomp = "N2", absolute = TRUE, 
    pres.amb = NULL, mol.f.name = NULL, vol.syr = NULL, cmethod = "removed", 
    imethod = "linear", extrap = FALSE, addt0 = TRUE, showt0 = TRUE, 
    dry = FALSE, std.message = TRUE, check = TRUE, temp.std = getOption("temp.std", 
        as.numeric(NA)), pres.std = getOption("pres.std", as.numeric(NA)), 
    unit.temp = getOption("unit.temp", "C"), unit.pres = getOption("unit.pres", 
        "atm")) 
NULL
\end{Soutput}
\end{Schunk}

Most of the arguments have default values, but to calculate CH$_4$ production we must provide values for at least  \texttt{dat} (we will use \texttt{vol}), \texttt{comp} (methane concentration), \texttt{temp} (biogas temperature), and \texttt{pres} (biogas pressure)\footnote{.      
  By default, temperature is in $^\circ$C and pressure in atm, but these can be changed in the function call with the \texttt{temp.unit} and \texttt{pres.unit} arguments, or globally with \texttt{options}. The same default values applies for temperature and pressure for presentation of biogas and methane, but these can be changed in the function call with the \texttt{temp.std} and \textttt{pres.std} argurments.}. along with the names of a few columns in the input data frames. If \texttt{temp} and/or \texttt{pres} arguments are not provided, biogas volumes will not be standardized.

