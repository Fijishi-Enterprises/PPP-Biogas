%% Next 2 lines needed for non-Sweave vignettes
%\VignetteEngine{knitr::knitr} 
%\VignetteIndexEntry{Calculating methane and biogas production and production rates using volumetric methods}

\documentclass{article}

%%\usepackage[version=3]{mhchem} %chemical formulas
\usepackage[colorlinks = true, urlcolor = blue]{hyperref} % Must be loaded as the last package

\begin{Schunk}
\begin{Sinput}
> library(knitr)
> #opts_chunk$set(cache=FALSE,tidy=FALSE,highlight=FALSE)
> opts_chunk$set(cache = FALSE, tidy = FALSE, fig.align = "center")
> library(biogas)
>   options(width=75)
\end{Sinput}
\end{Schunk}

\title{Calculating methane and biogas production and production rates using volumetric  methods}
\author{Nanna Løjborg and Sasha D. Hafner (\texttt{sasha.hafner@eng.au.dk})}

\usepackage{Sweave}
\begin{document}
\input{cumBgVol_function-concordance}

\maketitle

\section{Introduction}
Biochemical methane potential (BMP) has become an important number in the biogas industri, as it can reveal essential knowledge of several factors of concern when producing biogas, such as substrate and inoculum behavior, or more process related variables (temp., pres., stirring, etc.). It is commonly used to determine the methane potential and anaerobic biodegradability of a given substrate. 
BMP is most commonly evaluated by monitoring the production of biogas. This can be done in several ways including volumetric, manometric, gravimetric and gas density methods. In volumetric methods biogas volumes are measured under constant conditions from different techniques such as liquid displacement and syringe methods . In liquid displacement method, the biogas is moved to an external system measuring the volume. As biogas is produced it is collected in the external liquid displacement system where it displaces an amount of the barrier liquid, equivalent to the biogas volume .  In the syringe method, lubricated syringes are manually placed in the reactor of interest. Here, the pressure increase resulting from biogas production will force the piston inwards the syringe until the pressure equilibrates the surrounding (often atmospheric) pressure. Produced biogas volume is equivalent to the displacement of the piston and can be read directly off the syringe.   
We developed a function that evaluate volumetric measurements only, to simplify functions within the biogas package, making it serviceable for public less experinced users. 
This document provides a brief description of the volumetric biogas calculation function (cumBgVol) for new users.
We have assumed that readers are familiar with biogas data collection, the biogas package and R.

\section{Overview of the function}
cumBgVol() is a ''high-level'' function within the biogas package. The purpose of cumBgVol() is to convert volume data collected in the laboratory to cumulative biogas and CH$_4$ production and to calculate production rates. The function can handle data from any number of bottles. For simple operations (e.g. interpolation and standardization of biogas volume) cumBgVol() is supported by calls to external low-level functions (Table \ref{tab:externalfunctionsummary}). The low-level functions are straight-forward to use, and details can be found in their individual help files.
This document describes the use of cumBgVol(). 

\begin{table}[h!]
  \begin{center}
  \caption{Operations done with the low-level functions in cumBgVol(). All functions are vectorized. See help files for more details.}
  \label{tab:lowfunctionsummary}
  \vspace{3pt}
  
  \begin{tabular}{ll}
    \hline
    Operation                                    &   Function \\
    \hline
    Standardize gas volume                       &   \texttt{stdVol()} \\
    Interpolate composition etc.                 &   \texttt{interp()} \\
    Structurize and sort data                    &   \texttt{cumBgDataPrep()} \\   
		\hline
  \end{tabular}
  \end{center}
\end{table}

In general, cumBg* functions are compiled of four sections: check arguments, restructuring and sorting data, interpolation if needed, and biogas standardization and calculations. Restructuring and sorting data and interpolation are handled by the external functions interp() and cumBgDataPrep(), respectively. From interp() gas composition, cumulative biogas production, and other variables can be interpolated to a specified time if required. From cumBgDataPrep() 'wide' and 'long' data structure are restructured to 'longcombo' data, which is required for cumBgVol() to further calculate cumulative biogas and CH$_4$ production and production rates. Additionally, data is sorted, headspace is added if provided, and composition data is corrected if it seems to be a percentage. If data of concern is mixed (interval and cumulative, \texttt{empty.name = TRUE}), these will be sorted and biogas volume standardized within cumBgDataPrep() to obtain interval data only. Subsequently, the now restructured and sorted data is standardized in cumBgVol() by an external function called stdVol(), if not already standardized. 
Two methods are commonly used to evaluate volumetric biogas measurements. Method 1 is based on normalized CH4 concentrations, whereas method 2 accounts for the actual CH4 in the bottle headspace. Both methods are available through cumBgVol() and results is expected to be independent of method. The examples below describe cumulative biogas calculation by volumetric method 1 on three datasets of different structures (wide, long, and longcombo). 
All external functions are within the biogas package
##NTS: Might be beneficial to describe both methods. Ensure it is method 1 in all examples. Also, might make more sense to have longcombo as the first example, as this i default.

The arguments for the function are:

% Next line won't wrap correctly. Width doesn't affect args(). Done manually below by copying and editing output in .tex file.
\begin{Schunk}
\begin{Sinput}
>   #args(cumBgVol)
\end{Sinput}
\end{Schunk}

Most of the arguments have default values, but to calculate CH$_4$ production we must provide values for at least  \texttt{dat} (we will use \texttt{vol}), \texttt{comp} (methane concentration), \texttt{temp} (biogas temperature), and \texttt{pres} (biogas pressure)\footnote{.      
  By default, temperature is in $^\circ$C and pressure in atm, but these can be changed in the function call with the \texttt{temp.unit} and \texttt{pres.unit} arguments, or globally with \texttt{options}. The same default values applies for temperature and pressure for presentation of biogas and methane, but these can be changed in the function call with the \texttt{temp.std} and \textttt{pres.std} argurments.}. along with the names of a few columns in the input data frames. If \texttt{temp} and/or \texttt{pres} arguments are not provided, biogas volumes will not be standardized.
By default \texttt{interval = TRUE} and \texttt{data.struct} is set as 'longcombo'. Wide and long structured data will be restructedered to 'longcombo' internally by cumBgDataPrep(), when specified by the \texttt{data.struct} argument. When data are cumulative, the interval argument should be set to FALSE. When data are mixed interval and cumulative respons variables (\texttt{empty.name != TRUE}), the interval arguement is ignored. 
Similarly, there is an \texttt{id.name} argument for the reactor identification code (ID) column (default = "id"). The default value is \texttt{"id"}. For \texttt{data.struct = "wide"}, there is no ID columns. Instead data for each bottle, have individual columns and column names, which are used as ID codes. Here, the name of the column containing the response variables (\texttt{dat.name}), is set as the name of the first column with respons variables. All following columns are assumed to also have measurement data. 
Furthermore, we need to specify the name of the time column containing time data using the \texttt{time.name} argument (default = "time").
This name must be the same in both \texttt{dat} and \texttt{comp} data frames. Time data may be POSIX objects, but then t0 will not be added to rows by the cumBg* functions. In addition the \texttt{addt0} argument is used to add row with "time zero" (\texttt{time.name = 0}) for each reactor in order to calculate production rates for the first observation (default = TRUE). Whereas, \texttt{showt0} determines if the "time zero" should be returned in the output (default = TRUE if time.name is numeric and contains 0 and otherwise FALSE). 
The \texttt{comp.name} argument is used to indicate which column within the \texttt{comp} data frame contains the CH$_4$ content (as mole fraction in dry biogas, normalized so the sum of mole fractions of CH$_4$ and CO$_2$ sum to unity) (default = xCH4). \texttt{comp} may also just be a single value instead of a data frame or column. When providing a single value for \texttt{comp}, this value is extrapolated to all observations which pasifies the \texttt{comp.name} argument. Note that if \texttt{comp} argument is not provided, cumBg* funcitons will return calculations on biogas only and no calculations on CH$_4$. 
Additionally, a data frame containing headspace volumes may be accessible. A such data frame is generally optinal, but required if \texttt{imethod = "total"} (see description below) and should contain at least a headspace volume column (\texttt{vol.hs.name}) and a reactor identification column, with the same column name as \texttt{dat} and \texttt{comp} data frames. The headspace volume column can be set using the \texttt{vol.hs.name} argument (default = "vol.hs").
By default (\texttt{cmethod = "removed"}) the function calculates volumes following \cite{richards_methods_1991} as the product of standardized volume of biogas removed and normalized CH$_4$ content. If results should be based on the sum of methane removed and methane remaining in the reactor headspace, \texttt{cmethod} should be set to \texttt{"total"}. When \texttt{cmethod = "total"}, CH$_4$ concentration is calculated using all components (CH$_4$, CO$_2$, N$_2$, H$_2$S, etc.) instead of CH$_4$ and CO$_2$ only.
If any missing CH4 measurements, xCH4 is interpolated by the external "low-level" function \texttt{interp()}. Here, the \texttt{imethod} argument can be used to define interpolation method (default = "linear"), which is passed to interp().   
Similar, an \texttt{extrap} argument is passed to \texttt{interp()} (default = "FALSE"). The \texttt{extrap} argument is used to indicate if compositional data (\texttt{comp.name}) should be extrapolation (e.g. missing initial composition values).  
By default biogas is assumed to be saturated with water vapor. If biogas volumes are already standardized to dry conditions we need to set \texttt{dry = TRUE}.

\newpage
\section{Examples:calculation of cumulative production of biogas and CH$_4$ and production rates using a volumetric calculation method}
Calculation of cumulative biogas and CH$_4$ production and production rates, typically requires two data frames: Biogas quantity (volume measurements) and biogas composition (CH$_4$ fraction).
Input data may be structured in one of three ways: ``longcombo'', ``long'', and ``wide''. Default is "longcombo". All inputs are accepted, but the volumetric calculation methods within cumBgVol() only process ''longcombo'' data structure. ''wide'' and ''long'' data are restructured internally by the low-level function cumBgDataPrep(). In the following examples all three data structures will be addressed.

\subsection{'longcombo' data structure}
In this example, we will use the longcombo example data set included in the biogas package: \texttt{s3lcombo} for both biogas volumes and composition.
These data are from a BMP test carried out on three different substrates D, E, and F.
The experiment included three batch bottles: one for each substrate

# NTS: Do not have much information of these data yet. Maybe more info in GitHub.

\begin{Schunk}
\begin{Sinput}
> library(biogas)
> data("s3lcombo")
> dim(s3lcombo)
\end{Sinput}
\begin{Soutput}
[1] 21  4
\end{Soutput}
\begin{Sinput}
> head(s3lcombo)
\end{Sinput}
\begin{Soutput}
  id time.d vol.ml   xCH4
1  D 0.9438    103     NA
2  E 0.9451    106     NA
3  F 0.9472    107     NA
4  D 2.9060    192 0.6983
5  E 2.9090    181 0.6817
6  F 2.9100    203 0.6869
\end{Soutput}
\begin{Sinput}
> summary(s3lcombo)
\end{Sinput}
\begin{Soutput}
 id        time.d            vol.ml           xCH4       
 D:7   Min.   : 0.9438   Min.   :103.0   Min.   :0.6527  
 E:7   1st Qu.: 2.9100   1st Qu.:110.0   1st Qu.:0.6632  
 F:7   Median :10.0000   Median :136.0   Median :0.6800  
       Mean   :19.2421   Mean   :141.9   Mean   :0.6741  
       3rd Qu.:34.0100   3rd Qu.:181.0   3rd Qu.:0.6829  
       Max.   :57.8400   Max.   :203.0   Max.   :0.6983  
                                         NA's   :4       
\end{Soutput}
\end{Schunk}

\subsubsection*{Cumulative production}
Calculating cumulative biogas and CH$_4$ production and production rates, is the first step in processing data from a BMP trial. Subsequently, BMP can be calculated by the high-level function summBg() included in the biogas package.
Cumulative biogas and CH$_4$ production and production rates from volumetric data with \texttt{s3lcombo} data frame as the only input, can be calculated from \texttt{cumBgVol()}.

To calculate CH$_4$ production from these longcombo data, we must provide values for at least \texttt{dat} and \texttt{comp}, which is in a combined data frame (we will use \texttt{s3lcombo}), \texttt{temp} (biogas temperature), and \texttt{pres} (biogas pressure) along with the names of a few columns in our input data frame.
We can use the default values \texttt{"longcombo"}, \texttt{"id"}, and \texttt{"xCH4"} for the \texttt{data.struct}, \texttt{id.name}, and \texttt{comp.name} arguments, respectively. Whereas, the \texttt{dat.name} argument needs to be specified as \texttt{"vol.ml"}.
Similar, default values can be used for \texttt{cmethod = "removed"}, evaluating CH$_4$ concentration based on normalized CH$_4$ and CO$_2$ values, and for \texttt{imethod = "linear"}, resulting in internal linear interpolation of xCH4 by callin the \texttt{interp} function. 
Furthemore, initial composition (xCH4) values are missing in the \texttt{s3lcombo} data frame. Therefore, we need to set \texttt{extrap = TRUE} for extrapolation of these initial CH$_4$ values
By default biogas is assumed to be saturated with water vapor (\texttt{dry = FALSE}). Biogas volumes have not been standardized to dry conditions and hence, default value for the \texttt{dry} argument, matches requirements for making an actual evaluation of the dataset.
  
\begin{Schunk}
\begin{Sinput}
> cum.prod.lc <- cumBg(s3lcombo, temp = 25, pres = 1, data.struct = "longcombo", 
+                      time.name = 'time.d', 
+                      dat.name = 'vol.ml', 
+                      extrap = TRUE)
\end{Sinput}
\end{Schunk}

The output looks like this:

\begin{Schunk}
\begin{Sinput}
> head(cum.prod.lc)
\end{Sinput}
\begin{Soutput}
  id  time.d vol.ml   xCH4 temperature pressure       vBg      vCH4
1  D  0.0000     NA     NA          NA       NA   0.00000   0.00000
2  D  0.9438    103 0.6983          25        1  91.40334  63.91110
3  D  2.9060    192 0.6983          25        1 170.38293 119.13525
4  D  5.8860    141 0.6800          25        1 125.12497  85.20396
5  D 10.0000    112 0.6646          25        1  99.39004  66.15145
6  D 23.1000    200 0.6946          25        1 177.48222 123.44367
       cvBg    cvCH4     rvBg     rvCH4
1   0.00000   0.0000       NA        NA
2  91.40334  63.9111 96.84609 67.716783
3 261.78628 183.0464 86.83260 60.715143
4 386.91124 268.2503 41.98824 28.591933
5 486.30129 334.4018 24.15898 16.079593
6 663.78351 457.8454 13.54826  9.423181
\end{Soutput}
\begin{Sinput}
> dim(cum.prod.lc)
\end{Sinput}
\begin{Soutput}
[1] 24 12
\end{Soutput}
\end{Schunk}

The data frame that is returned has maintained the longcombo structure with all the original columns in \texttt{vol} and contains additional columns from volumetric biogas calculations. 

In these columns, \texttt{v} stands for (standardized) volume, \texttt{cv} (standardized) cumulative volume, \texttt{rv} stands for (standardized) volume production rate, and \texttt{Bg} and \texttt{CH4} for biogas and methane.
So \texttt{cvBg} contains standardized cumulative biogas production and \texttt{cvCH4} contains standardized cumulative CH$_4$ production.

It is probably easier to understand the data in the output graphically.
Here we'll use the \texttt{ggplot} function from the \texttt{ggplot2} package to plot it.

\begin{Schunk}
\begin{Sinput}
> library(ggplot2)
> ggplot.longcombo <- ggplot(cum.prod.lc, aes(time.d, cvCH4, colour = factor(id))) + 
+   geom_point() +
+   geom_line(aes(group = id)) +
+   labs(x = "Time [d]", y = "Cumulative methane production  [mL]", colour = "Bottle ID")  + 
+   theme_bw() 
> plot(ggplot.longcombo)
\end{Sinput}
\end{Schunk}

\newpage
\subsection{'wide' data structure}
In this example, we will use a wide structured example dataset included in the biogas package, having the data frame \texttt{feedVol} for biogas volumes and \texttt{feedSetup} for bottle identification and substrate and inoculum masses. The \texttt{feedsetup} column will not be relevant when using \texttt{cumBgVol}, but is required for further calculation of BMP using \texttt{summBg}.
BMP measurement data are from a batch test carried out on animal feed ingredients along with cellulose as a control. 
The experiment included 12 batch bottles:
\begin{itemize}
  \item Three bottles with inoculum only (BK)
  \item Three bottles with cellulose and inoculum (CEL)
  \item Three bottles with animal feed ingredient 1 and inoculum (SC)
  \item Three bottles with animal feed ingredient 2 and inoculum (SD)
\end{itemize}

A typical automated volumetric method called AMPTS II was used to measure biogas production: an online, standardized lab-measurement platform for BMP tests. Applying AMPTS II, the measured volumes are standardized and the composition is 100\% methane. Therefore, the \texttt{comp}) argument is set to 1 when calling the cumBgVol function. Furthermore, pressure is set to a fixed value (atmospheric) and temperature to 0°C. 
\begin{Schunk}
\begin{Sinput}
> data("feedVol")
> dim(feedVol)