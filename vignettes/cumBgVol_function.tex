%% Next 2 lines needed for non-Sweave vignettes
%\VignetteEngine{knitr::knitr} 
%\VignetteIndexEntry{Calculating methane and biogas production and production rates using volumetric methods}

\documentclass{article}

%%\usepackage[version=3]{mhchem} %chemical formulas
\usepackage[colorlinks = true, urlcolor = blue]{hyperref} % Must be loaded as the last package

\begin{Schunk}
\begin{Sinput}
> library(knitr)
> #opts_chunk$set(cache=FALSE,tidy=FALSE,highlight=FALSE)
> opts_chunk$set(cache = FALSE, tidy = FALSE, fig.align = "center")
> library(biogas)
>   options(width=75)
\end{Sinput}
\end{Schunk}

\title{Calculating methane and biogas production and production rates using volumetric  methods}
\author{Nanna Løjborg and Sasha D. Hafner (\texttt{sasha.hafner@eng.au.dk})}

\usepackage{Sweave}
\begin{document}
\input{cumBgVol_function-concordance}

\maketitle

\section{Introduction}
Biochemical methane potential (BMP) has become an important number in the biogas industri, as it can reveal essential knowledge of several factors of concern when producing biogas, such as substrate and inoculum behavior, or more process related variables (temp., pres., stirring, etc.). It is commonly used to determine the methane potential and biodegradability of a given substrate. A newly developed biogas package address issues with time-consuming calculations and lack of reproducible among laboratories for obtaining BMP (ref. biogas package or help file). The biogas package consists of ten function including cumBg(), which is used to calculate cumulative production of biogas and methane (CH$_4$) and production rates with either volumetric, manometric, gravimetric or gas density methods. These production values and rates can be furhter used to calculate BMP. cumBg() is a large and rather complicated function, which requires some proficiency in R for use. ##NTS: might be preferable to avoid including cumBg(), as these new functions should replace cumBg() at some point. 
We developed a function that only evaluate volumetric measurements to simplify functions within the biogas package, making it serviceable for public less experinced users. 
This document provides a brief description of the volumetric biogas calculation function (cumBgVol) for new users.
We have assumed that readers are familiar with biogas data collection, the biogas package and R.

\section{Overview of the function}
cumBgVol() is a ''high-level'' function within the biogas package. The purpose of cumBgVol() is to convert volume data collected in the laboratory to cumulative biogas and CH$_4$ production and to calculate production rates. The function can handle data from any number of bottles. For simple operations (e.g. interpolation and standardization of biogas volume) cumBgVol() is supported by calls to external low-level functions (Table \ref{tab:externalfunctionsummary}). The low-level functions are straight-forward to use, and details can be found in their individual help files.
This document describes the use of cumBgVol(). 

\begin{table}[h!]
  \begin{center}
  \caption{Operations done with the low-level functions in cumBgVol(). All functions are vectorized. See help files for more details.}
  \label{tab:lowfunctionsummary}
  \vspace{3pt}
  
  \begin{tabular}{ll}
    \hline
    Operation                                    &   Function \\
    \hline
    Standardize gas volume                       &   \texttt{stdVol()} \\
    Interpolate composition etc.                 &   \texttt{interp()} \\
    Structurize and sort data                    &   \texttt{cumBgDataPrep()} \\   
		\hline
  \end{tabular}
  \end{center}
\end{table}

In general, cumBg* functions are compiled of four sections: check arguments, restructuring and sorting data, interpolation if needed, and biogas standardization and calculations. Restructuring and sorting data and interpolation are handled by the external functions interp() and cumBgDataPrep(), respectively. From interp() gas composition, cumulative biogas production, and other variables can be interpolated to a specified time if required. From cumBgDataPrep() 'wide' and 'long' data structure are restructured to 'longcombo' data, which is required for cumBgVol() to further calculate cumulative biogas and CH$_4$ production and production rates. Additionally, data is sorted, headspace is added if provided, and composition data is corrected if it seems to be a percentage. If data of concern are mixed (interval and cumulative, \texttt{empty.name = TRUE}), these will be sorted and biogas volume standardized within cumBgDataPrep() to obtain interval data only. Subsequently, the now restructured and sorted data is standardized in cumBgVol() by an external function called stdVol(), if not already standardized. 
Two methods are commonly used to evaluate volumetric biogas measurements. Method 1 is based on normalized CH4 concentrations, whereas method 2 accounts for the actual CH4 in the bottle headspace. Both methods are available through cumBgVol() and results is expected to be independent of method. The examples below describe cumulative biogas calculation by volumetric method 1 on three datasets of different structures (wide, long, and longcombo). 
All external functions are within the biogas package
##NTS: Might be beneficial to describe both methods. Ensure it is method 1 in all examples. Also, might make more sense to have longcombo as the first example, as this i default.

\section{Examples:calculation of cumulative production of biogas and CH$_4$ and production rates using a volumetric calculation method}
Calculation of cumulative biogas and CH$_4$ production and production rates, typically requires two data frames: Biogas quantity (volume measurements) and biogas composition (CH$_4$ fraction).
Input data may be structured in one of three ways: ``long'', ``wide'', or ``longcombo''. Default is "longcombo". All inputs are accepted, but the volumetric calculation methods within cumBgVol() only process ''longcombo'' data structure. ''wide'' and ''long'' data are restructured internally by the low-level function cumBgDataPrep(). In the following examples all three data structures will be addressed.  

\newpage
\subsection{'wide' data structure}
In this example, we will use a wide structured example dataset included in the biogas package, having the data frame \texttt{feedVol} for biogas volumes. BMP measurement data are from a batch test carried out on animal feed ingredients along with cellulose as a control. 
The experiment included 12 batch bottles:
\begin{itemize}
  \item Three bottles with inoculum only (BK)
  \item Three bottles with cellulose and inoculum (CEL)
  \item Three bottles with animal feed ingredient 1 and inoculum (SC)
  \item Three bottles with animal feed ingredient 2 and inoculum (SD)
\end{itemize}

A typical automated volumetric method called AMPTS II was used to measure biogas production: an online, standardized lab-measurement platform for BMP tests. Applying AMPTS II, the measured volumes are standardized and the composition is 100\% methane. Therefore, the \texttt{comp}) argument is set to 1 when calling the cumBgVol function. Furthermore, pressure is set to a fixed value (atmospheric) and temperature to 0°C. 


\subsubsection*{Cumulative production}
Calculating cumulative biogas and CH$_4$ production and production rates, is the first step in processing data from a BMP trial. Subsequently, BMP can be calculated by the high-level function summBg() included in the biogas package.
Cumulative biogas and CH$_4$ production and production rates from volumetric data with \texttt{feedVol} data frame as input, can be calculated from \texttt{cumBgVol()}.
The arguments for the function are:


Most of the arguments have default values, but to calculate CH$_4$ production we must provide values for at least  \texttt{dat} (we will use \texttt{vol}), \texttt{comp} (we will set a single constant composition value}), \texttt{temp} (biogas temperature), and \texttt{pres} (biogas pressure)\footnote{.      
  By default, temperature is in $^\circ$C and pressure in atm, but these can be changed in the function call with the \texttt{temp.unit} and \texttt{pres.unit} arguments, or globally with \texttt{options}.
}, along with the names of a few columns in our input data frames. 
By default \texttt{data.struct} is set as 'longcombo'. Wide and long structured data will be restructedered to 'longcombo' internally by cumBgDataPrep(), when specified by the \texttt{data.struct} argument. 
Furthermore, we need to specify the name of the time column in \texttt{feedVol} using the \texttt{time.name} argument.
This name must be the same in both data frames.
Similarly, there is an \texttt{id.name} argument for the reactor identification code (ID) column. For \texttt{data.struct = 'wide'}, there are no ID columns. Instead data for each bottle, has individual columns and column names, which is used as ID codes. Here, the name of the column containing the response variables (\texttt{dat.name}), is set as the name of the first column with respons variables (volume measurement). In this example observations are numbered 1 to 12 and hence, we will set \texttt{dat.name} to 1. All following columns are assumed to also have measurement data (volume). 
The \texttt{comp.name} argument is used to indicate which column within the \texttt{comp} data frame contains the CH$_4$ content (as mole fraction in dry biogas, normalized so the sum of mole fractions of CH$_4$ and CO$_2$ sum to unity). We have provided a single value for \texttt{comp}, meaning that the \texttt{comp.name} argument is not required.
By default (\texttt{cmethod = "removed"}) the function calculates volumes following \cite{richards_methods_1991} as the product of standardized volume of biogas removed and normalized CH$_4$ content. 
By default biogas is assumed to be saturated with water vapor. For AMPTS II data, biogas volume are already standardized to dry conditions. Therefore, we need to set \texttt{dry = TRUE}.


Note the message about standard temperature and pressure--it is important to make sure these values are correct, therefore users are reminded by a message\footnote{
  Remember that standard conditions can be set in the function call with \texttt{temp.std} and \texttt{pres.std}, or globally with \texttt{options()}. 
}. Also, note the message about applying single composition value to all observations. When defining pressure as a single value, pressure is assumed to be constant and the same for all observations.  

The output looks like this:



The data frame that is returned has been restructured to longcombo structure and has all the original columns in \texttt{vol}, plus additional columns from the volumetric calculation method. 
In these columns, \texttt{v} stands for (standardized) volume, \texttt{cv} (standardized) cumulative volume, \texttt{rv} stands for (standardized) volume production rate, and \texttt{Bg} and \texttt{CH4} for biogas and methane.
So \texttt{cvBg} contains standardized cumulative biogas production and \texttt{cvCH4} contains standardized cumulative CH$_4$ production.

It is probably easier to understand the data in the output graphically.
Here we'll use the \texttt{ggplot} function from the \texttt{ggplot2} package to plot it.



\newpage
\subsection{'long' data structure}
In this example, we will use the example data sets included with the package: \texttt{vol} for biogas volumes and \texttt{comp} for composition.
These data are from a BMP test that was carried out on two different substrates A and B, and cellulose included as a ``control''.
The experiment included 12 batch bottles:
\begin{itemize}
  \item 3 bottles with substrate A and inoculum
  \item 3 bottles with substrate B and inoculum
  \item 3 bottles with cellulose and inoculum
  \item 3 bottles with inoculum only
\end{itemize}
Reactors consisted of 500 mL or 1.0 L glass bottles, and were sealed with a butyl rubber septum and a screw cap.
Initial substrate and inoculum masses were determined.
A typical volumetric method was used to measure biogas production: accumulated biogas was measured and removed intermittently using syringes, and composition was measured for some of these samples. 

\begin{Schunk}
\begin{Sinput}
> library(biogas)
> data("vol")
> dim(vol)
\end{Sinput}
\begin{Soutput}
[1] 288   4
\end{Soutput}
\begin{Sinput}
> head(vol)
\end{Sinput}
\begin{Soutput}
   id           date.time days vol
1 2_1 2014-06-07 13:00:00 1.98 393
2 2_1 2014-06-08 13:00:00 2.98 260
3 2_1 2014-06-09 13:00:00 3.98 245
4 2_1 2014-06-10 13:00:00 4.98 225
5 2_1 2014-06-11 13:00:00 5.98 200
6 2_1 2014-06-12 14:00:00 7.02 175
\end{Soutput}
\begin{Sinput}
> summary(vol)
\end{Sinput}
\begin{Soutput}
       id        date.time                        days       
 2_1    : 24   Min.   :2014-06-07 13:00:00   Min.   :  1.98  
 2_2    : 24   1st Qu.:2014-06-14 02:00:00   1st Qu.:  8.52  
 2_3    : 24   Median :2014-06-28 12:00:00   Median : 22.94  
 2_4    : 24   Mean   :2014-07-16 21:29:22   Mean   : 41.33  
 2_5    : 24   3rd Qu.:2014-07-26 04:45:00   3rd Qu.: 50.63  
 2_6    : 24   Max.   :2014-12-19 10:30:00   Max.   :196.92  
 (Other):144                                                 
      vol       
 Min.   : 98.0  
 1st Qu.:171.5  
 Median :225.0  
 Mean   :271.7  
 3rd Qu.:300.0  
 Max.   :840.0  
\end{Soutput}
\end{Schunk}

\begin{Schunk}
\begin{Sinput}
> data("comp")
> dim(comp)
\end{Sinput}
\begin{Soutput}
[1] 132   4
\end{Soutput}
\begin{Sinput}
> head(comp)
\end{Sinput}
\begin{Soutput}
     id           date.time  days      xCH4
516 2_1 2014-06-12 14:00:00  7.02 0.7104731
519 2_1 2014-06-19 14:00:00 14.02 0.7024937
522 2_1 2014-06-26 11:00:00 20.90 0.6659919
524 2_1 2014-07-03 10:00:00 27.85 0.6789466
525 2_1 2014-07-10 09:00:00 34.81 0.6951429
528 2_1 2014-07-24 10:00:00 48.85 0.6693053
\end{Soutput}
\begin{Sinput}
> summary(comp)
\end{Sinput}
\begin{Soutput}
       id       date.time                        days       
 2_1    :11   Min.   :2014-06-12 14:00:00   Min.   :  7.02  
 2_2    :11   1st Qu.:2014-06-26 11:00:00   1st Qu.: 20.90  
 2_3    :11   Median :2014-07-24 10:00:00   Median : 48.85  
 2_4    :11   Mean   :2014-07-31 13:47:43   Mean   : 56.01  
 2_5    :11   3rd Qu.:2014-08-28 10:00:00   3rd Qu.: 83.85  
 2_6    :11   Max.   :2014-10-13 13:00:00   Max.   :129.98  
 (Other):66                                                 
      xCH4       
 Min.   :0.5647  
 1st Qu.:0.6393  
 Median :0.6598  
 Mean   :0.6587  
 3rd Qu.:0.6786  
 Max.   :0.7115  
\end{Soutput}
\end{Schunk}

\subsubsection*{Cumulative production}
As for animal feed ingredients data, cumulative production of biogas and CH$_4$ and production rates are essential values to estimate prior to evaluating BMP.
Again, we can do this with the \texttt{cumBgVol()} function, using \texttt{vol} and \texttt{comp} data frames as input.
The required arguments for this function are the following:

To calculate CH$_4$ production from these long structured data, we must provide values for at least \texttt{dat} (we will use \texttt{vol}), \texttt{comp} (we will use \texttt{comp}), \texttt{data.struct} (in this case 'long'), \texttt{temp} (biogas temperature), and \texttt{pres} (biogas pressure)\footnote{
  By default, temperature is in $^\circ$C and pressure in atm, but these can be changed in the function call with the \texttt{temp.unit} and \texttt{pres.unit} arguments, or globally with \texttt{options}.
}, along with the names of a few columns in our input data frames. 
We need to specify the name of the time column in \texttt{vol} and \texttt{comp} using the \texttt{time.name} argument.
This name must be the same in both data frames.
Similarly, there is an \texttt{id.name} argument for the bottle ID column (used to match up volume and composition data). We can use the default value (\texttt{"id"}) because it matches the column name in \texttt{vol} and \texttt{comp}.
And, to indicate which column within the \texttt{comp} data frame contains the CH$_4$ content, we can use the default value (\texttt{"xCH4"}) for the \texttt{comp.name} argument, because it matches the column name in \texttt{comp}. 
Furthermore, the name of the column that contains the response variable in the \texttt{dat} data frame can be specified with the \texttt{dat.name} argument.
Here too we can use the default (\texttt{"vol"} for volumetric measurements.
By default (\texttt{cmethod = "removed"}) the function calculates volumes following \cite{richards_methods_1991} as the product of standardized volume of biogas removed and normalized CH$_4$ content.
In addition to interpolation for later observations, an extrapolation argument (\texttt{extrap}) can be provided if required. We do have initial biogas composition (compare the heads of the \texttt{vol} and \texttt{comp} data frames) so will need to extrapolate, in addition to interpolation for later observations. Therefore, we need to set \texttt{extrap = TRUE}.
  
\begin{Schunk}
\begin{Sinput}
> cum.prod.l <- cumBgVol(vol, comp = comp, time.name = "days", temp = 35, pres = 1, 
+ 		  extrap = TRUE)